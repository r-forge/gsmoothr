\documentclass{article}

\usepackage{amsmath}
\usepackage{amscd}
\usepackage[tableposition=top]{caption}
\usepackage{ifthen}
\usepackage[utf8]{inputenc}
\topmargin 0in
\headheight 0in
\headsep 0in
\oddsidemargin 0in
\evensidemargin 0in
\textwidth 176mm
\textheight 215mm


\usepackage{Sweave}
\begin{document}

%\VignetteIndexEntry{Using FIRMAGene -- Finding Isoforms using Robust Multichip Analysis for (Affymetrix) Gene platforms}

\title{\texttt{FIRMAGene}: Finding Isoforms using Robust Multichip Analysis for (Affymetrix) Gene platforms}
\author{Mark Robinson \\ \texttt{mrobinson@wehi.edu.au}}
\maketitle

\section{Introduction}

\noindent This document gives a brief introduction to the \texttt{FIRMAGene} package, which is designed to detect {\em differential} splicing events using the Affymetrix Gene 1.0 ST platform.  On this platform, there are probes for each known exon along the gene, but not necessarily the (generally) 4 probes per probe selection region (PSR) that the Exon 1.0 ST chip gives.  After robustly fitting the RMA linear model, a persistence of non-zero residuals (all in the same direction) are often evidence of alternative splice forms.  We illustrate a standard processing of the data for 

\section{An analysis of the Affymetrix public tissue panel dataset}

Using the R \texttt{aroma.affymetrix} package, we process the data and get access to the residuals.  If you are new to using \texttt{aroma.affymetrix}, there are installation and setup instructions at:

\texttt{http://groups.google.com/group/aroma-affymetrix/}

Here, we use the Affymetrix 33-sample (11-tissue) dataset to illustrate some tissue-specific alternative splicing candidates.

First, we preprocess the data.

\begin{Schunk}
\begin{Sinput}
> library(aroma.affymetrix)
\end{Sinput}
\begin{Soutput}
R.cache v0.1.7 (2008-02-27) successfully loaded. See ?R.cache for help.
R.rsp v0.3.4 (2008-03-06) successfully loaded. See ?R.rsp for help.
 Type browseRsp() to open the RSP main menu in your browser.
aroma.apd v0.1.3 (2006-06-14) successfully loaded. See ?aroma.apd for help.
Patching /Users/mrobinson/.Rpatches/aroma.affymetrix/20090318/AffymetrixCdfFile.computeAffinities.R 
Patching /Users/mrobinson/.Rpatches/aroma.affymetrix/20090318/AffymetrixCelFile.BG.R 
Patching /Users/mrobinson/.Rpatches/aroma.affymetrix/20090318/AffymetrixCelSet.BG.R 
Patching /Users/mrobinson/.Rpatches/aroma.affymetrix/20090318/AffymetrixCelSet.convertToUnique.R 
Patching /Users/mrobinson/.Rpatches/aroma.affymetrix/20090318/AffymetrixCelSet.cpgBoxplots.R 
Patching /Users/mrobinson/.Rpatches/aroma.affymetrix/20090318/AffymetrixCelSet.writeSgr.R 
Patching /Users/mrobinson/.Rpatches/aroma.affymetrix/20090318/AromaCellCpgFile.R 
Patching /Users/mrobinson/.Rpatches/aroma.affymetrix/20090318/MatSmoothing.R 
Patching /Users/mrobinson/.Rpatches/aroma.affymetrix/20090318/bpmapCluster2Cdf.R 
\end{Soutput}
\begin{Sinput}
> cdf <- AffymetrixCdfFile$fromChipType("HuGene-1_0-st-v1", verbose = -1)